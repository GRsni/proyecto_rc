\documentclass[]{article}
\usepackage[spanish]{babel}
\usepackage{graphicx}
\usepackage{xcolor}
\usepackage[utf8]{inputenc}
\usepackage{fancyhdr}
\usepackage{lastpage}
\usepackage{enumitem}
\usepackage{listings}
\usepackage{float}
\usepackage{verbatim}
\usepackage{booktabs}
\usepackage{makecell}
\usepackage{tabularx}

\pagestyle{fancy}
\fancyhf{}
\rfoot{Page \thepage\hspace{1pt} de~\pageref{LastPage}}

\title{Proyecto de Redes de Computadores}
\author{Guillermo López García
\and
Gonzalo Ulibarri García
\and
Félix Lázaro Palacio
\and
Alfredo Ramos García
\and
Santiago Jesús Mas Peña
\and
\ldots
\and
\ldots
\and
\ldots
}

\begin{document}
\maketitle
\newpage

\section{Primera Hoja}
\begin{table}[h!]
  \begin{center}
    \caption{Preguntas Integrantes}
    \begin{tabular}{ccccc}
      \toprule
      \textbf{Integrantes} & \textbf{Pregunta 1} & \textbf{Pregunta 2}
      \textbf{Pregunta 3} & \textbf{Pregunta 4} \\
      \midrule
      1 &  &  \\
      2 &  &  \\
      3 &  &  \\
      4 &  &  \\
      5 &  &  \\
      6 &  &  \\
      7 &  &  \\
      8 &  &  \\
      \bottomrule
    \end{tabular}
  \end{center}
\end{table}

\begin{table}[h!]
  \begin{center}
    \caption{Preguntas Integrantes}
    \begin{tabular}{ccccc}
      \toprule
      \textbf{Integrantes} & \textbf{Pregunta 5} & \textbf{Pregunta 6}
      \textbf{Pregunta 7} & \textbf{Pregunta 8} \\
      \midrule
      1 &  &  \\
      2 &  &  \\
      3 &  &  \\
      4 &  &  \\
      5 &  &  \\
      6 &  &  \\
      7 &  &  \\
      8 &  &  \\
      \bottomrule
    \end{tabular}
  \end{center}
\end{table}

\textbf{Coordinador/a: } Guillermo López García. \newline
\textbf{Ponente 1: } Guillermo López García. \newline
\textbf{Ponente 2: } Guillermo López García.

\underline{Pregunta 1:}
\underline{Pregunta 2:}
\underline{Pregunta 3:}
\underline{Pregunta 4:}
\underline{Pregunta 5:}
\underline{Pregunta 6:}
\underline{Pregunta 7:}
\underline{Pregunta 8:}

\newpage

\section{Segunda Hoja}
\begin{table}[h!]
  \centering
  \begin{center}
    \caption{Tabla profesor}
    \resizebox{\columnwidth}{!}{
        \begin{tabular}{ccc}
          \toprule
          \textbf{Conceptos a valorar} & \textbf{Puntuación Máxima} & \textbf{Puntuación Otorgada} \\
          \midrule
          Contenido del dossier claro y detallado & 3 puntos &  \\
          \midrule
          Maquetación/Formato del dossier         & 1 punto  &  \\
          \midrule
          Ponente 1                               & 1 punto  &  \\
          \midrule
          Ponente 2                               & 1 punto  &  \\
          \midrule
          Preguntas formuladas                    & 4 puntos &  \\
          \midrule
          Puntuación total                        & \multicolumn{2}{c}{} \\
          \bottomrule
        \end{tabular}
    }
  \end{center}
\end{table}

\newpage

\tableofcontents

\newpage

\section{Documento 1: Planos de Cableado Horizontal}

\newpage

\section{Documento 2: Distribuidores}
\begin{table}[ht!]
  \centering
  \begin{center}
    \caption{Tabla distribuidores 1}
    \resizebox{\columnwidth}{!}{
        \begin{tabular}{ccccccccc}
          \toprule
          \multicolumn{3}{l}{Etiqueta del distribuidor: A- } & \multicolumn{2}{c}{} \\
          \midrule
          \multicolumn{3}{l}{Altura mínima del distribuidor: 27U } & \multicolumn{2}{c}{} \\
          \midrule
          \multicolumn{3}{l}{Ubicación: Primera planta } & \multicolumn{2}{c}{} \\
          \midrule
          \textbf{Dispositivo} & \textbf{Capa OSI} & \textbf{Altura} &
          \textbf{Nº Puertos} & \textbf{Estándar*} & \textbf{TAT Etiquetas*} &
          \textbf{Tipo Conector} & \textbf{Categoría} & \textbf{Cantidad} \\
          \midrule
          Switch & Enlace & 1U & 48 & IEEE 802 & RJ1-RJ139(Fija \- Datos) PA1-PA7 & G45 / RJ45 LC & 7 & 4 \\
          \midrule
           &  &  \\
          \midrule
           &  &  \\
          \midrule
           &  &  \\
          \midrule
           & \multicolumn{2}{c}{} \\
          \bottomrule
        \end{tabular}
    }
  \end{center}
\end{table}

\begin{table}[ht!]
  \centering
  \begin{center}
    \caption{Tabla distribuidores 2}
    \resizebox{\columnwidth}{!}{
        \begin{tabular}{ccccccccc}
          \toprule
          \multicolumn{3}{l}{Etiqueta del distribuidor: B- } & \multicolumn{2}{c}{} \\
          \midrule
          \multicolumn{3}{l}{Altura mínima del distribuidor: 11U } & \multicolumn{2}{c}{} \\
          \midrule
          \multicolumn{3}{l}{Ubicación: Planta baja } & \multicolumn{2}{c}{} \\
          \midrule
          \textbf{Dispositivo} & \textbf{Capa OSI} & \textbf{Altura} &
          \textbf{Nº Puertos} & \textbf{Estándar*} & \textbf{TAT Etiquetas*} &
          \textbf{Tipo Conector} & \textbf{Categoría} & \textbf{Cantidad} \\
          \midrule
          Router & Red & 1U & 2GEn4\-10GE & IEEE 1588 & \- & G45 / RJ45 LC & 7 & 4 \\
          \midrule
           &  &  \\
          \midrule
           &  &  \\
          \midrule
           &  &  \\
          \midrule
           &  &  \\
          \midrule
           & \multicolumn{2}{c}{} \\
          \bottomrule
        \end{tabular}
    }
  \end{center}
\end{table}

\newpage

\section{Documento 3: Plano de Cableado Vertical}

\newpage

\section{Documento 4: Plano de Conexión}

\newpage

\section{Documento 5: Justificadores}


% Para listar fragmentos de codigo de un lenguaje de programacion
% \lstset{language=, texcl=true}
% \begin{lstlisting}[frame=single]
% \end{lstlisting}

% Para listar una secuencia de elementos
% \textbf{}
% \begin{enumerate}
    % \item
% \end{enumerate}

% Para mostrar una figura
% \begin{figure}[H]
% \centering
% \includegraphics[width=0.7\linewidth]{}
% \caption{}
% \end{figure}

% Para listar el contenido de un archivo de texto
% \verbatiminput{.txt}

\end{document}
