\documentclass[]{article}
\usepackage[spanish]{babel}
\usepackage{graphicx}
\usepackage{xcolor}
\usepackage[utf8]{inputenc}
\usepackage{fancyhdr}
\usepackage{lastpage}
\usepackage{enumitem}
\usepackage{listings}
\usepackage{float}
\usepackage{verbatim}
\usepackage{booktabs}
\usepackage{makecell}
\usepackage{tabularx}

\pagestyle{fancy}
\fancyhf{}
\rfoot{Page \thepage\hspace{1pt} de~\pageref{LastPage}}

\title{Proyecto de Redes de Computadores}
\author{Guillermo López García
\and
Gonzalo Ulibarri García
\and
Félix Lázaro Palacio
\and
Alfredo Ramos García
\and
Santiago Jesús Mas Peña
\and
\ldots
\and
\ldots
\and
\ldots
}

\begin{document}
\maketitle
\newpage

\section{Primera Hoja}
\begin{table}[h!]
  \begin{center}
    \caption{Preguntas Integrantes}
    \begin{tabular}{ccccc}
      \toprule
      \textbf{Integrantes} & \textbf{Pregunta 1} & \textbf{Pregunta 2}
      \textbf{Pregunta 3} & \textbf{Pregunta 4} \\
      \midrule
      1 &  &  \\
      2 &  &  \\
      3 &  &  \\
      4 &  &  \\
      5 &  &  \\
      6 &  &  \\
      7 &  &  \\
      8 &  &  \\
      \bottomrule
    \end{tabular}
  \end{center}
\end{table}

\begin{table}[h!]
  \begin{center}
    \caption{Preguntas Integrantes}
    \begin{tabular}{ccccc}
      \toprule
      \textbf{Integrantes} & \textbf{Pregunta 5} & \textbf{Pregunta 6}
      \textbf{Pregunta 7} & \textbf{Pregunta 8} \\
      \midrule
      1 &  &  \\
      2 &  &  \\
      3 &  &  \\
      4 &  &  \\
      5 &  &  \\
      6 &  &  \\
      7 &  &  \\
      8 &  &  \\
      \bottomrule
    \end{tabular}
  \end{center}
\end{table}

\textbf{Coordinador/a: } Guillermo López García. \newline
\textbf{Ponente 1: } Guillermo López García. \newline
\textbf{Ponente 2: } Gonzalo Ulibarri García.

\underline{Pregunta 1:}
\underline{Pregunta 2:}
\underline{Pregunta 3:}
\underline{Pregunta 4:}
\underline{Pregunta 5:}
\underline{Pregunta 6:}
\underline{Pregunta 7:}
\underline{Pregunta 8:}

\newpage

\section{Segunda Hoja}
\begin{table}[h!]
  \centering
  \begin{center}
    \caption{Tabla profesor}
    \resizebox{\columnwidth}{!}{
        \begin{tabular}{ccc}
          \toprule
          \textbf{Conceptos a valorar} & \textbf{Puntuación Máxima} & \textbf{Puntuación Otorgada} \\
          \midrule
          Contenido del dossier claro y detallado & 3 puntos &  \\
          \midrule
          Maquetación/Formato del dossier         & 1 punto  &  \\
          \midrule
          Ponente 1                               & 1 punto  &  \\
          \midrule
          Ponente 2                               & 1 punto  &  \\
          \midrule
          Preguntas formuladas                    & 4 puntos &  \\
          \midrule
          Puntuación total                        & \multicolumn{2}{c}{} \\
          \bottomrule
        \end{tabular}
    }
  \end{center}
\end{table}

\newpage

\tableofcontents

\newpage

\section{Documento 1: Planos de Cableado Horizontal}

\newpage

\section{Documento 2: Distribuidores}
\begin{table}[ht!]
  \centering
  \begin{center}
    \caption{Tabla distribuidores 1}
    \resizebox{\columnwidth}{!}{
        \begin{tabular}{ccccccccc}
          \toprule
          \multicolumn{3}{l}{Etiqueta del distribuidor: A- } & \multicolumn{2}{c}{} \\
          \midrule
          \multicolumn{3}{l}{Altura mínima del distribuidor: 27U } & \multicolumn{2}{c}{} \\
          \midrule
          \multicolumn{3}{l}{Ubicación: Primera planta } & \multicolumn{2}{c}{} \\
          \midrule
          \textbf{Dispositivo} & \textbf{Capa OSI} & \textbf{Altura} &
          \textbf{Nº Puertos} & \textbf{Estándar*} & \textbf{TAT Etiquetas*} &
          \textbf{Tipo Conector} & \textbf{Categoría} & \textbf{Cantidad} \\
          \midrule
          Switch Fibra & Enlace & 1U & 48 PoE / 4 SFP & IEEE 802 & RJA1-RJA132(Fija --- Datos) PAA1-PAA7 & G45 / RJ45 LC & 7 & 4 \\
          \midrule
          Punto de Acceso & Enlace & --- & 1 PoE & IEEE 802.3af & PAA1-PAA7 & LC & 7 & 7 \\
          \midrule
          Path Panel Datos/VoIP/Wifi & Física & 2U & 48 & --- & RJA1-RJA132(Fija --- Datos) PAA1-PAA7 & G45/RJ45 & --- & 4 \\
          \midrule
           &  &  \\
          \midrule
           & \multicolumn{2}{c}{} \\
          \bottomrule
        \end{tabular}
    }
  \end{center}
\end{table}

\begin{table}[ht!]
  \centering
  \begin{center}
    \caption{Tabla distribuidores 2}
    \resizebox{\columnwidth}{!}{
        \begin{tabular}{ccccccccc}
          \toprule
          \multicolumn{3}{l}{Etiqueta del distribuidor: B- } & \multicolumn{2}{c}{} \\
          \midrule
          \multicolumn{3}{l}{Altura mínima del distribuidor: 27U } & \multicolumn{2}{c}{} \\
          \midrule
          \multicolumn{3}{l}{Ubicación: Planta baja } & \multicolumn{2}{c}{} \\
          \midrule
          \textbf{Dispositivo} & \textbf{Capa OSI} & \textbf{Altura} &
          \textbf{Nº Puertos} & \textbf{Estándar*} & \textbf{TAT Etiquetas*} &
          \textbf{Tipo Conector} & \textbf{Categoría} & \textbf{Cantidad} \\
          \midrule
          Router & Red & 1U & 2GE/n4\---10GE & IEEE 1588 & --- & G45 / RJ45 LC & 7 & 1 \\
          \midrule
          Switch Fibra & Enlace & 1U & 48 PoE / 4 SFP & IEEE 802 & RJB1-RJB232(Fija --- Datos) PAB1-PAB7 & G45 / RJ45 LC & 7 & 6 \\
          \midrule
          Punto de Acceso & Enlace & --- & 1 PoE & IEEE 802.3af & PAB1-PAB7 & LC & 7 & 7 \\
          \midrule
          Path Panel Datos/VoIP/Wifi & Física & 2U & 48 & --- & RJB1-RJB232(Fija --- Datos) PAB1-PAB7 & G45/RJ45 & --- & 4 \\
          \midrule
          Path Panel Fibra & Física & 2U & 12 & --- & FB1 --- FB12(Multimode VDSL/ADSL+) & LC & --- & 1 \\
          \midrule
          Dispositivo Firewall & Transporte & 1U & 7 (5 + 2 USB) & --- & --- & RJ45 / USB & --- & 1 \\
          \midrule
           &  &  \\
          \midrule
           &  &  \\
          \midrule
           & \multicolumn{2}{c}{} \\
          \bottomrule
        \end{tabular}
    }
  \end{center}
\end{table}

\begin{table}[ht!]
  \centering
  \begin{center}
    \caption{Direccionamiento Planta Alta}
    \resizebox{\columnwidth}{!}{
        \begin{tabular}{cc}
          \toprule
          \textbf{Elemento} & \textbf{Dirección IP} \\
          \midrule
          VLAN Switch 1 --- VLAN Switch 4 & 172.20.0.2/22 --- 172.20.0.5/22 \\
          RJA1 --- RJA132 (Fija --- Datos) & 172.20.0.6/22 --- 172.20.0.137/22 \\
          \midrule
          Punto de Acceso (Etiquetas) \\
          PAA1 --- PAA7 & 172.20.0.138/22 --- 172.20.0.144/22 \\
          \bottomrule
        \end{tabular}
    }
  \end{center}
\end{table}

\begin{table}[ht!]
  \centering
  \begin{center}
    \caption{Direccionamiento Planta Baja}
    \resizebox{\columnwidth}{!}{
        \begin{tabular}{cc}
          \toprule
          \textbf{Elemento} & \textbf{Dirección IP} \\
          \midrule
          VLAN Switch 1 --- VLAN Switch 6 & 172.20.2.2/22 --- 172.20.2.5/22 \\
          RJB1 --- RJB132 (Fija --- Datos) & 172.20.2.6/22 --- 172.20.2.237/22 \\
          \midrule
          Router: \\
          IP Privada: & 172.20.2.1/22 \\
          IP Pública: & 10.0.0.1/24 \\
          \midrule
          Punto de Acceso (Etiquetas) \\
          PAB1 --- PAB7 & 172.20.2.238/22 --- 172.20.2.244/22 \\
          \bottomrule
        \end{tabular}
    }
  \end{center}
\end{table}
\newpage

\section{Documento 3: Plano de Cableado Vertical}

\newpage

\section{Documento 4: Plano de Conexión}

\newpage

\section{Documento 5: Justificadores}

\begin{enumerate}[label=\alph*]
   \item \underline{Respecto al Documento 1}:
	Respecto al cableado horizontal, se han instalado tomas de conexiones de tamaño específico para cada posible puesto de trabajo. El número de conexiones varía atendiendo al número de dispositivos máximos conectados simultáneamente que se tenga previsto. Se ha considerado también añadir tomas de conexiones para posibles puntos de acceso que cubran toda la superficie del edificio, colocándolas en punto estratégicos que cumplan este requisito. Las tomas dedicadas para los puntos de acceso tendrán únicamente un puerto para estos. \\
	Para los racks, se ha  considerado como buena localización en la planta baja el espacio bajo la escalera hacia la primera planta, y en ésta, la denominada "sala de informática", ya que probablemente en esta sala se encuentren más racks para uso como servidor de datos o para lo que sea pertinente. \\
	La distribución del cableado horizontal se ha pensado para que sea lo más centralizada posible, recorriendo el hall del edificio y manteniendo distancias no muy prolongadas entre las tomas de conexiones y el cableado del hall, permitiendo así que con un único rack por planta ésta quede cubierta.
   \item \underline{Respecto al Documento 2}:
   \item \underline{Respecto al Documento 3}:
        Hemos considerado la interconexión entre los switches con fibra óptica para aumentar el rendimiento y la velocidad. Cada switch de cada planta irá conectado con fibra multimodo al siguiente switch del respectivo rack. No haremos distinciones entre conexiones, ya sean para voz o datos. \\
        Pensamos además conectar los puntos de acceso con Power Ethernet (PoE) para darle un suministro de energía a dicho dispositivos. \\
        Para las etiquetas, hemos realizado una codificación simple, separando solamente las etiquetas correspondientes a los puntos de acceso. Dicha codificación será:
        \begin{center} (tipo de conector) (etiq distribuidor) (número de conexión)\end{center}
        \begin{center}Ejemplo: RJA4 (conexión 4, planta superior datos / VoIP)\end{center}
        Para el routing, utilizaremos solo un router en el Rack B (Planta Baja). Esto es debido a que no es necesario amplificar la señal de las conexiones mas lejanas, ya que, en ningún caso superara los 100 metros.
   \item \underline{Respecto al Documento 4}:
   \item \underline{Conclusión Final}:
\end{enumerate}

% Para listar fragmentos de codigo de un lenguaje de programacion
% \lstset{language=, texcl=true}
% \begin{lstlisting}[frame=single]
% \end{lstlisting}

% Para listar una secuencia de elementos
% \textbf{}
% \begin{enumerate}
    % \item
% \end{enumerate}

% Para mostrar una figura
% \begin{figure}[H]
% \centering
% \includegraphics[width=0.7\linewidth]{}
% \caption{}
% \end{figure}

% Para listar el contenido de un archivo de texto
% \verbatiminput{.txt}

\end{document}
